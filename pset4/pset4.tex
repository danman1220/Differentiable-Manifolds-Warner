%%%%%%%%%%%%%%%%%%%%%%%%%%%%%%%%%%%%%%%%%
% Short Sectioned Assignment
% LaTeX Template
% Version 1.0 (5/5/12)
%
% This template has been downloaded from:
% http://www.LaTeXTemplates.com
%
% Original author:
% Frits Wenneker (http://www.howtotex.com)
%
% License:
% CC BY-NC-SA 3.0 (http://creativecommons.org/licenses/by-nc-sa/3.0/)
%
%%%%%%%%%%%%%%%%%%%%%%%%%%%%%%%%%%%%%%%%%

%----------------------------------------------------------------------------------------
%    PACKAGES AND OTHER DOCUMENT CONFIGURATIONS
%----------------------------------------------------------------------------------------

\documentclass[paper=a4, fontsize=11pt]{scrartcl} % A4 paper and 11pt font size

\usepackage[T1]{fontenc} % Use 8-bit encoding that has 256 glyphs
\usepackage{fourier} % Use the Adobe Utopia font for the document - comment this line to return to the LaTeX default
\usepackage[english]{babel} % English language/hyphenation
\usepackage{amsmath,amsfonts,amsthm} % Math packages
\usepackage{mathrsfs}

\usepackage{lipsum} % Used for inserting dummy 'Lorem ipsum' text into the template

\usepackage{sectsty} % Allows customizing section commands
\allsectionsfont{\centering \normalfont\scshape} % Make all sections centered, the default font and small caps

\usepackage{fancyhdr} % Custom headers and footers
\pagestyle{fancyplain} % Makes all pages in the document conform to the custom headers and footers
\fancyhead{} % No page header - if you want one, create it in the same way as the footers below
\fancyfoot[L]{} % Empty left footer
\fancyfoot[C]{} % Empty center footer
\fancyfoot[R]{\thepage} % Page numbering for right footer
\renewcommand{\headrulewidth}{0pt} % Remove header underlines
\renewcommand{\footrulewidth}{0pt} % Remove footer underlines
\setlength{\headheight}{13.6pt} % Customize the height of the header

\numberwithin{equation}{section} % Number equations within sections (i.e. 1.1, 1.2, 2.1, 2.2 instead of 1, 2, 3, 4)
\numberwithin{figure}{section} % Number figures within sections (i.e. 1.1, 1.2, 2.1, 2.2 instead of 1, 2, 3, 4)
\numberwithin{table}{section} % Number tables within sections (i.e. 1.1, 1.2, 2.1, 2.2 instead of 1, 2, 3, 4)

\setlength\parindent{0pt} % Removes all indentation from paragraphs - comment this line for an assignment with lots of text

%----------------------------------------------------------------------------------------
%    TITLE SECTION
%----------------------------------------------------------------------------------------

\newcommand{\horrule}[1]{\rule{\linewidth}{#1}} % Create horizontal rule command with 1 argument of height

\title{    
\normalfont \normalsize 
\textsc{Differentiable Manifolds} \\ [25pt] % Your university, school and/or department name(s)
\horrule{0.5pt} \\[0.4cm] % Thin top horizontal rule
\huge Problem Set 4: 1.9 \\ % The assignment title
\horrule{2pt} \\[0.5cm] % Thick bottom horizontal rule
}

\author{Joshua Ramette \& Daniel Halmrast} % Your name

\date{\normalsize\today} % Today's date or a custom date

\begin{document}

\maketitle % Print the title

%----------------------------------------------------------------------------------------
%    PROBLEM 1
%----------------------------------------------------------------------------------------

\section*{Problem 1.8}

Problem 1.8:\\
Obtain the classical implicit function theorem from the general implicit function theorem stated in theorem 1.38.\\


Theorem 1.38: Assume that $\psi:M^c \to N^d$ is $C^{\infty}$, that $n \in N$, $P=\psi^{-1}(n)$ is nonempty, and that $d\psi$ is surjective for all $m \in P$. 
Then, $P$ has a unique manifold structure such that $(P,i)$ is a submanifold of $M$, and $i$ is an imbedding, and $\mathrm{dim}(P)=c-d$. \\

Let $U \subset \mathbb{R}^{c-d} \times \mathbb{R}^d$ and $f: \mathbb{R}^c \to \mathbb{R}^d$. Furthermore, the jacobian is nonsingular in the first $d$ dimensions at a point $(r_0, s_0)$ such that $f(r_0, s_0) = 0$. Because of the continuity of the determinant of the first $d$ dimensions of the Jacobian, there exists a rectangular neighborhood $V_0 \times W_0 \subset U$ of $(r_0, s_0)$ such that that first $d$ dimensions of the Jacobian is nonsingular on this neighborhood, implying that the Jacobian is surjective. Then by 1.38, $P = f^{-1} |_{V_0 \times W_0}(0)$ is an imbedded submanifold of $V_0 \times W_0$ of dimension $c-d$. \\

Now we construct $\pi_1: V_0 \times W_0 \to V_0$ the projection function, and $\pi_1 \circ \iota$ is $C^{\infty}$.  Then, $d(\pi_1 \circ \iota): T(P)_{(r_0,s_0)} \to T(V_0)_{r_0}$ is an isomorphism, and it follows from Theorem 1.30  that there exists an open set $(r_0,s_0) \in V \times W \subset V_0 \times W_0$, such that $(\pi_1 \circ \iota) ^{-1}|_{P \cap V \times W}: V \to P \cap V \times W$ is bijective, $C^{\infty}$. \\

Then, consider $g = (\pi_2 \circ \iota) \circ (\pi_1 \circ \iota)^{-1} |_{P \cap V \times W}: V \to W$, which satisfies the requirements of the implitict function theorem. Let $(p,q) \in V \times W$. Then, $f(p,q) = 0 \implies (p,q) \in P$, and then $g(p) = (\pi_2 \circ \iota) \circ (\pi_1 \circ \iota)^{-1} (p) = \pi_2 \circ \iota (p,q) = q $ for the same unique $q$ because of the bijectivity of $(\pi_1 \circ \iota)^{-1}$. \\

In instead we have $q = g(p)$, then $f(p, g(p)) = 0$ since $(p, g(p)) \in P$ because of the restriction built into the $g$ function.




\section*{Problem 1.9}

Problem 1.9:\\
Let $f:\mathbb{R^2} \to \mathbb{R}$ be defined by
\[
f(x,y) = x^3 + xy + y^3 + 1
\]

For which points $p \in \{(0,0),(\frac{1}{3},\frac{1}{3}),(\frac{-1}{3},\frac{-1}{3})\}$
is $f^{-1}({f(p)})$ an imbedded submanifold?\\
\\


Solution:\\
The general Jacobian for this transformation is $\big[3x^2 + y, 3y^2 + x\big]$, which will be nonsurjective
iff both components are zero.
Solving for this condition yields two solutions: The Jacobian is zero if $x=y=0$ and if $x=y=\frac{-1}{3}$\\

Consider first $p=(0,0)$. The resulting "level curve", $P(0,0)=f^{-1}({f(0,0)})$ contains the point $(0,0)$.
Since this point satisfies the condition for the Jacobian to be nonsurjective, the Jacobian is not everywhere nonsurjective 
for $P(0,0)$.\\

Consider as well the point $p=(\frac{1}{3},\frac{1}{3})$. 
Does the resulting level curve $V(x^3 + xy + y^3 - \frac{5}{27}$ contain either $(0,0)$ or $(\frac{-1}{3},\frac{-1}{3})$?
It is simple to check that it does not contain either point, Therefore, this level curve is a imbedded submanifold.\\

Consider as well the point $p=(\frac{-1}{3},\frac{-1}{3})$. 
Obviously, it's level curve contains $(\frac{-1}{3},\frac{-1}{3})$. Therefore, the Jacobian is not everywhere nonsurjective, and the level curve is not an imbedded submanifold.

\section*{Problem 1.10}

Problem 1.10:\\

Let $M$ be a compact manifold of dimension $n$, and let $f: M \to \mathbb{R}^n$ be $C^{\infty}.$ Prove that $f$ cannot be everywhere non-singular. \\

$M$ compact implies that $f(M)$ is also compact, which also implies it is closed, and bounded in $\mathbb{R}^n$ by Heine Borel. Thus, $f$ attains a maximum on $M$ at some point $x \in M$. But then the first derivatives with respect to a coordinate chart containing $x$ must all be zero, implying the Jacobian at that point is singular, thus $f$ cannot be everywhere nonsingular.





%----------------------------------------------------------------------------------------

\end{document}
