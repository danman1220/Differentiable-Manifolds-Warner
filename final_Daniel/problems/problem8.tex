\section{The Story of Manifolds}

\emph{Problem 8}: Tell the story of differentiable manifolds.
\\
\emph{Solution}:
The study of topology allows mathematics to talk about more general spaces than
just $\R^n$. In particular, there is a certain class of spaces that seem locally
similar to Euclidean space, but globally differ in some meaningful way. (Think
of $S^2$, which is locally homeomorphic to some subset of $\R^2$, but is
a compact space, unlike $\R^2$) Euclidean space has a rich
structure of differentiability that allows calculus to be done on it, and it
seems natural to ask if such a structure can also be given to more general
topological spaces.
\par
To that end, we define a differentiable manifold as a
topological space that is locally Euclidean, along with a series of "co-ordinate
systems" on open subsets of the space that satisfy the properties of a
differentiable structure. In particular, the co-ordinate systems need to cover
the whole space, they need to match up on overlaps in a $C^{\infty}$ way, and
the set of co-ordinate maps must be maximal in the sense that it contains all
possible compatible maps.
\par
With such a structure, it becomes possible to talk about generalized
derivatives. By looking at linear derivations of germs of functions from the
manifold to $\R$, we develop the idea of tangent vectors, which locally look
like directional derivatives. Furthermore, by looking at wedge products of
(duals of) these tangent vectors, we can develop a more general theory of
differential forms on the manifold, which allow for integration over subsets of
the manifold.
\par
In the category of differentiable manifolds, the objects are the manifolds, and
the arrows are $C^{\infty}$ maps (in the sense that they preserve the
$C^{\infty}$ness of the coordinates). Then, one can define a functor, known as
exterior differentiation, from this category to the category of forms on
differentiable manifolds.
\par
With these tools, we define integration of a differential form (which, in this
case can be thought of as a generalized multidimensional function, vector field,
etc.) on a manifold by locally "pulling back" the form to normal Euclidean
space, and integrating the form in the usual way in $\R^N$.
\par
Not that the theory of manifolds is all analysis, though.
Homology and cohomology information can be pulled from the
manifold. For cohomology, the algebraic spaces are the spaces of p-dimensional
forms on the manifold, and the arrows are exterior differentiation. For
homology, the spaces are free combinations of maps of Euclidean shapes into the
manifold, with the arrows being the boundary operation. From this, we get the
powerful result that the qth differential cohomology of a manifold is isomorphic 
to the dual of the qth simplicial homology, a result known as deRham's theorem.
\par
All in all, the theory of differentiable manifolds is a way to use local
Euclidean-ness of certain topological spaces to develop calculus and algebra on
those spaces.
