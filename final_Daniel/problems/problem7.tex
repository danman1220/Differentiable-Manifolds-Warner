\section{Integration on Manifolds}

\emph{Problem 7}: Give your favorite example of an integral on manifolds
computation.
\\

\emph{Solution}: For this problem, we will compute the integral
\[
\int_{S^1}\frac{xdy -ydx}{x^2+y^2}
\]
For this integral, we will use the co-ordinate maps
\[
\begin{aligned}
x &= r\cos(\theta)\\
y &= r\sin(\theta)
\end{aligned}
\]
which, by setting r to 1 and taking the exterior derivative, yields the
identities
\[
\begin{aligned}
dx &= -\sin(\theta)d\theta\\
dy &=  \cos(\theta)d\theta
\end{aligned}
\]
Then, the integral becomes
\[
\begin{aligned}
\int_{S^1} (\sin^2(\theta) + \cos^2(\theta))d\theta &= \int_{S^1} d\theta\\
                                             &= 2\pi
\end{aligned}
\]
What makes this example interesting is the fact that the form integrated is
actually a closed form on the space $\R^2\setminus\{0\}$. Furthermore, since the
integral is nonzero, the form is not exact. This demonstrates that there exist
closed 1-forms on $S^1$ that are not exact, which means that the first
cohomology group of $S^1$ is nontrivial. Since $S^1$ is a typical example of a
simple space that differs from Euclidean space, this form gives an example of
something that breaks the triviality of the cohomology of Euclidean space. In
particular, it suggests that singularities (like the one this form has at zero)
have something to do with cohomology (and homology) breaking. 
