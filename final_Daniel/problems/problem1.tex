\section{The Torus Skew Line}

\emph{Problem 1}: The torus is the manifold $S^1 \times S^1$. Consider $S^1$ as 
the unit circle in the complex plane. We define $\phi: \R \to S^1 \times S^1$ by

\begin{center}
$\phi(t) = (e^{2 \pi it}, e^{2 \pi i\alpha t})$
\end{center}

where $\alpha$ is an irrational number. Prove that $(\R, \phi)$ is a dense
submanifold of $S^1 \times S^1$.
\\
\emph{Solution}:
This proof will be done in two parts. First I will show that $(\R, \phi)$ is in
fact a submanifold of the torus, then I will show that it is dense in the torus.


\subsection*{$(\R,\phi)$ is a submanifold of $S^1 \times S^1$}

A submanifold is a pair $(X, f)$ of a manifold $X$ and an immersion $f$ that is
one-to-one. $\R$ is already known to be a manifold, so it suffices to show that
$\phi$ is a one-to-one immersion.
\\
%-------------
%  IMMERSION |
%-------------
In order for $\phi$ to be an immersion, $d\phi_x$ must be nonsingular
$\forall x \in \R$. Since there is a global coordinate system on both the torus
and $\R$, it suffices to check nonsingularity in these coordinate systems. To do
so, $d\phi$ must be known. So,
\begin{center}
\[
\begin{aligned}
d\phi_x: &\R_x \to \mathbb{T}^2_{\phi(x)}              \\
d\phi_x\left(\frac{\partial}{\partial t}_x \right)     &=
        \frac{\partial(e^{2 \pi it})}{\partial t}_x
        \frac{\partial}{\partial y_1}_{\phi(x)} +
        \frac{\partial(e^{2 \pi i\alpha t})}{\partial t}_x
        \frac{\partial}{\partial y_2}_{\phi(x)}        \\
      &=2\pi ite^{2 \pi it}
        \frac{\partial}{\partial y_1}_{\phi(x)} +
        2\pi i \alpha te^{2\pi i\alpha t}
        \frac{\partial}{\partial y_2}_{\phi(x)}        \\
      &=2\pi it\left(e^{2 \pi it}
        \frac{\partial}{\partial y_1}_{\phi(x)} +
        \alpha e^{2\pi i\alpha t}
        \frac{\partial}{\partial y_2}_{\phi(x)}\right) \\
\end{aligned}
\]
\end{center}
In order for this to be nonsingular, both the
$\frac{\partial}{\partial y_1}_{\phi(x)}$ coordinate and the
$\frac{\partial}{\partial y_2}_{\phi(x)}$ coordinate must not both be zero.
Such a condition occurs when $e^{2\pi it}=\alpha e^{2\pi i \alpha t} = 0$.
But, this implies that
\[
\begin{aligned}
    \exists n_1, n_2 \in \Z, \\
       t        &= n_1 \\
       \alpha t &= n_2 \\
    &\implies \alpha t - t \in \Z \\
    &\implies \alpha - 1 \in \Z
\end{aligned}
\]
But this violates the assumption that $\alpha$ is irrational. So, $d\phi$ must
be nonsingular everywhere.
\\
%------------
% INJECTION |
%------------
Next, $\phi$ must be an injective map.
But, $\phi$ is easily seen to be nonsingular, by a similar argument to the one for
$d\phi$. Thus, $\phi$ must be injective.

\subsection*{$\phi(\R)$ Is Dense In $\mathbb{T}^2$}

For the skew line to be dense in the torus, it must be that every open
set in $\mathbb{T}^2$ intersects $\phi(\R)$.
This will be shown by showing that any point in $\mathbb{T}^2$ is arbitrarily
close to the skew line.
\\
To do this, I will view $\mathbb{T}^2$ as $\R^2\big/ \Z\times\Z$, and my maps as
$\phi(t)\to[t,\alpha t]$. Here, $(x_1,y_1) \sim (x_2,y_2) \iff (x_2-x_1,y_2-y_1)
\in \Z \times \Z$.
\par
Consider a point $[x,y]$. To show the skew line is dense in $\mathbb{T}^2$, I
will show that it intersects the region $\{[x,y]|y\in(a,b)\}$, where $a$ and $b$
are chosen as the endpoints of the $\frac{1}{2^n}$ partition of the unit
interval that contains $y$.
\par
Suppose ther exists an $n$ such that the skew line does not intersect the box.
Then, by the integer translational symmetry of this quotient space, the skew
line does not intersect any of the other partitions of the interval. Thus, the
skew line must be passing through the boundaries of the partition.
\\
However, there are only finitely many boundaries, so by the pidgeonhole
principle, there must be at least one for which the skew line passes through
twice. This violates the assumption that the skew line imbedding was one-to-one,
and so it must not be the case that the skew line does not intersect the
neighborhood of $[x,y]$. Therefore, the skew line is dense in $\mathbb{T}^2$.


