%%%%%%%%%%%%%%%%%%%%%%%%%%%%%%%%%%%%%%%%%
% Short Sectioned Assignment
% LaTeX Template
% Version 1.0 (5/5/12)
%
% This template has been downloaded from:
% http://www.LaTeXTemplates.com
%
% Original author:
% Frits Wenneker (http://www.howtotex.com)
%
% License:
% CC BY-NC-SA 3.0 (http://creativecommons.org/licenses/by-nc-sa/3.0/)
%
%%%%%%%%%%%%%%%%%%%%%%%%%%%%%%%%%%%%%%%%%

%----------------------------------------------------------------------------------------
%    PACKAGES AND OTHER DOCUMENT CONFIGURATIONS
%----------------------------------------------------------------------------------------

\documentclass[paper=a4, fontsize=11pt]{scrartcl} % A4 paper and 11pt font size

\usepackage[T1]{fontenc} % Use 8-bit encoding that has 256 glyphs
\usepackage{fourier} % Use the Adobe Utopia font for the document - comment this line to return to the LaTeX default
\usepackage[english]{babel} % English language/hyphenation
\usepackage{amsmath,amsfonts,amsthm} % Math packages
\usepackage{mathrsfs}
\usepackage{tikz-cd} %commutative diagrams package

\usepackage{lipsum} % Used for inserting dummy 'Lorem ipsum' text into the template

\usepackage{sectsty} % Allows customizing section commands
\allsectionsfont{\centering \normalfont\scshape} % Make all sections centered, the default font and small caps

\usepackage{fancyhdr} % Custom headers and footers
\pagestyle{fancyplain} % Makes all pages in the document conform to the custom headers and footers
\fancyhead{} % No page header - if you want one, create it in the same way as the footers below
\fancyfoot[L]{} % Empty left footer
\fancyfoot[C]{} % Empty center footer
\fancyfoot[R]{\thepage} % Page numbering for right footer
\renewcommand{\headrulewidth}{0pt} % Remove header underlines
\renewcommand{\footrulewidth}{0pt} % Remove footer underlines
\setlength{\headheight}{13.6pt} % Customize the height of the header

\numberwithin{equation}{section} % Number equations within sections (i.e. 1.1, 1.2, 2.1, 2.2 instead of 1, 2, 3, 4)
\numberwithin{figure}{section} % Number figures within sections (i.e. 1.1, 1.2, 2.1, 2.2 instead of 1, 2, 3, 4)
\numberwithin{table}{section} % Number tables within sections (i.e. 1.1, 1.2, 2.1, 2.2 instead of 1, 2, 3, 4)

\setlength\parindent{0pt} % Removes all indentation from paragraphs - comment this line for an assignment with lots of text

%----------------------------------------------------------------------------------------
%    TITLE SECTION
%----------------------------------------------------------------------------------------

\newcommand{\horrule}[1]{\rule{\linewidth}{#1}} % Create horizontal rule command with 1 argument of height

\title{
\normalfont \normalsize 
\textsc{Differentiable Manifolds} \\ [25pt] % Your university, school and/or department name(s)
\horrule{0.5pt} \\[0.4cm] % Thin top horizontal rule
\huge Problem Set 5: 2.1, 2.2, 2.3 \\ % The assignment title
\horrule{2pt} \\[0.5cm] % Thick bottom horizontal rule
}

\author{Joshua Ramette \& Daniel Halmrast} % Your name

\date{\normalsize\today} % Today's date or a custom date

\begin{document}

\maketitle % Print the title

%----------------------------------------------------------------------------------------
%    PROBLEM 1
%----------------------------------------------------------------------------------------

\section*{Problem 2.1}
\subsection*{Part a}

Let U, V, W be vector spaces, with $\phi: V \times W \to V \otimes W$ the natural mapping, $ l: V \times W \to U$ bilinear. \\

NTS: exists unique $\widetilde{l}: V \otimes W \to U$ such that $\widetilde{l} \circ \phi = l$.\\

Define $\widetilde{l}$ on decomposable tensors of the form $v \otimes w$ as $\widetilde{l}(v \otimes w) = l(v,w)$ 
and extend to all of $V \otimes W$ by linearity. \\

It is clear that $\widetilde{l} \circ \phi(v, w) $ = $\widetilde{l} (v \otimes w) $ = $l(v,w)$ and the diagram commutes.

Uniqueness: Suppose $\widetilde{l}'$ is another linear lifting of $l$. 
Then, for $(v_0, w_0)$, 
$\widetilde{l} \circ \phi(v_0, w_0) = \widetilde{l}(v_0 \otimes w_0) = l(v_0,w_0) = \widetilde{l}' \circ \phi(v_0,w_0) = \widetilde{l}'(v_0 \otimes w_0)$,
and thus $\widetilde{l}' = \widetilde{l}$. \\

Now to prove isomorphism.
The universal mapping property can be summarized in a commutative diagram:
For some $A$, $B$ is said to satisfy the universal mapping property if $\forall C \forall l, \exists !\widetilde{l}$
such that the following diagram commutes.\\
\begin{center}
\begin{tikzcd}
B \arrow[dr, dashed, "\widetilde{l}"] \\
A \arrow[r, "l"] \arrow[u, "\phi_B"]
    & C
\end{tikzcd}
\end{center}

To prove uniqueness, let $(X, \phi_X)$ be another object that satisfies the mapping property for $A$.
Then, by applying the mapping property of $B$ to $X$, we get the following diagram.
\begin{center}
\begin{tikzcd}
B \arrow[dr, dashed, "\widetilde{\phi_X}"] \\
A \arrow[r, "\phi_X"] \arrow[u, "\phi_B"] \arrow[d, "\phi_B"]
    & X \arrow[dl, dashed, "\widetilde{\phi_B}"]\\
B
\end{tikzcd}
\end{center}

Then, from the diagram, since $\widetilde{\phi_X} \circ \phi_B =\phi_X$ and $\widetilde{\phi_B} \circ \phi_X = \phi_B$,
it follows that $\phi_X = \widetilde{\phi_X} \circ \widetilde{\phi_B} \circ \phi_X$
and $\phi_B = \widetilde{\phi_B} \circ \widetilde{\phi_X} \circ \phi_B$.
Thus, $\widetilde{\phi_B}$ and $\widetilde{\phi_X}$ are inverses of each other that compose to the identity,
and form an isomorphism of $X$ and $B$.


\subsection*{Part b}

$V \otimes W \cong W \otimes V$. 
Define the isomorphism as, for $\psi: V \times W \to W \times V$ the canonical isomorphism, $\psi_0: V \otimes W \to W \otimes V$. \\

Let $\phi$ be the bilinear map from part (a) of $V \times W$ into $V \otimes W$ and $\phi'$ the bilinear map of $W \times V$ into $W \otimes V$. 
Then, $ \psi_0 = \phi' \circ \psi$, with natural inverse $\psi_0 ^{-1} = \phi \circ \psi^{-1}$ 
where $\psi_0$ is extended to all of $V \otimes W$ via linearity. \\

\subsection*{Part c}

$U \otimes (V \otimes W) = (U \otimes V) \otimes W$. Apply the same lifting as (b) on $\psi: U \times (V \times W) \to (U \times V) \times W$.\\

\subsection*{Part d}

$\alpha$ is injective by linearity $\alpha(v_1) - \alpha(v_2) = 0 \to \alpha(v_1 - v_2) = 0$ and triviality of the kernel. \\

Let $T: V \to W$ be an element of Hom$(V,W)$. $T(x_i) = \sum c_j y_j = w_i$. 
Then, $T(V) = T(\sum c_i y_i) = \sum c_i T(x_i)  = \sum c_i (\sum(c_j y_j))  = \sum_i w_i $. Let $f_i =  \pi_i$ be the $i$-th coordinate projection. 
Then $T(V) = \sum f_i(v) w_i = \sum\alpha(f_i \otimes w_i)(v) = \alpha(\sum(f_i \otimes w_i)(v).$ Then $\alpha$ is surjective as well. \\

\subsection*{Part e}

Suppose $(v \otimes w) \in V \otimes W)$.
Then $(v \otimes w) = (\sum c_i e_i) \otimes (\sum d_j f_j) = \sum_i((c_i e_i) \otimes (\sum d_j f_j)) =  \sum_ic_i (e_i \otimes (\sum d_j f_j)) =  \sum_i \sum_j c_i (e_i \otimes (d_j f_j)) = \sum_i \sum_j c_i d_j (e_i \otimes f_j)$. 
Thus the desired set is a basis.


%----------------------------------------------------------------------------------------

%----------------------------------------------------------------------------------------
%    PROBLEM 2
%----------------------------------------------------------------------------------------

\section*{Problem 2.2}
\subsection*{Part a}

Provide an example of a homogeneous tensor that is not decomposable

\begin{proof}
Let $V$ be a vector space, and $V \otimes V$ the corrseponding tensor product space.
Furthermore, let $v, w$ be vectors in $V$.
Then, the tensor $v \otimes w + w \otimes v$ is homogeneous of degree two, but is not decomposable.
\end{proof}

\subsection*{Part b}

Show that for $dim(V) \leq 3$, every homogeneous element of $\Lambda (V)$ is decomposable.

\begin{proof}

Let $V$ be a three dimensional vector space with basis $\{v_1, v_2, v_3\}$.
Then, the corresponding exterior algebra has basis elements
\\
\begin{center}
\begin{tabular}{c c c}
                & $v_1 \wedge v_2 \wedge v_3$ &\\
$v_1 \wedge v_2$  & $v_1 \wedge v_3$  & $v_2 \wedge v_3$\\
$v_1$             & $v_2$               & $v_3$\\
                  & $1$ &
\end{tabular}
\end{center}

It suffices to check for degree two elements of $\Lambda(V)$ that they are decomposable.

To this end, let $c_1 v_1 \wedge v_2 + c_2 v_1 \wedge v_3 + c_3 v_2 \wedge v_3$ be an arbitrary
degree two element of the exterior algebra.
Then, it is easy to see that 
\[
\begin{aligned}
c_1 v_1 \wedge v_2 + c_2 v_1 \wedge v_3 + c_3 v_2 \wedge v_3 & = v_1 \wedge (c_1 v_2 + c_2 v_3) + c_3 v_2 \wedge v_3\\
& = (v_1 -\frac{c_1}{c_3}v_3) \wedge (c_1v_2 + c_2v_3)
\end{aligned}
\]

\end{proof}

\subsection*{Part c}

Give an example of a homogeneous indecomposable element of $\Lambda(V)$.

\begin{proof}
The element $v_1 \wedge v_2 + v_3 \wedge v_4$ for linearly independent $v_1...v_4$ is indecomposable.
\end{proof}

\subsection*{Part d}

Is $\alpha \wedge \alpha = 0$?

\begin{proof}
Since $\alpha \wedge \alpha = -\alpha \wedge \alpha$, this implies $\alpha \wedge \alpha =0$.
\end{proof}

\end{document}
