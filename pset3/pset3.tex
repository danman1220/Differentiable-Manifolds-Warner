%%%%%%%%%%%%%%%%%%%%%%%%%%%%%%%%%%%%%%%%%
% Short Sectioned Assignment
% LaTeX Template
% Version 1.0 (5/5/12)
%
% This template has been downloaded from:
% http://www.LaTeXTemplates.com
%
% Original author:
% Frits Wenneker (http://www.howtotex.com)
%
% License:
% CC BY-NC-SA 3.0 (http://creativecommons.org/licenses/by-nc-sa/3.0/)
%
%%%%%%%%%%%%%%%%%%%%%%%%%%%%%%%%%%%%%%%%%

%----------------------------------------------------------------------------------------
%    PACKAGES AND OTHER DOCUMENT CONFIGURATIONS
%----------------------------------------------------------------------------------------

\documentclass[paper=a4, fontsize=11pt]{scrartcl} % A4 paper and 11pt font size

\usepackage[T1]{fontenc} % Use 8-bit encoding that has 256 glyphs
\usepackage{fourier} % Use the Adobe Utopia font for the document - comment this line to return to the LaTeX default
\usepackage[english]{babel} % English language/hyphenation
\usepackage{amsmath,amsfonts,amsthm} % Math packages
\usepackage{mathrsfs}

\usepackage{lipsum} % Used for inserting dummy 'Lorem ipsum' text into the template

\usepackage{sectsty} % Allows customizing section commands
\allsectionsfont{\centering \normalfont\scshape} % Make all sections centered, the default font and small caps

\usepackage{fancyhdr} % Custom headers and footers
\pagestyle{fancyplain} % Makes all pages in the document conform to the custom headers and footers
\fancyhead{} % No page header - if you want one, create it in the same way as the footers below
\fancyfoot[L]{} % Empty left footer
\fancyfoot[C]{} % Empty center footer
\fancyfoot[R]{\thepage} % Page numbering for right footer
\renewcommand{\headrulewidth}{0pt} % Remove header underlines
\renewcommand{\footrulewidth}{0pt} % Remove footer underlines
\setlength{\headheight}{13.6pt} % Customize the height of the header

\numberwithin{equation}{section} % Number equations within sections (i.e. 1.1, 1.2, 2.1, 2.2 instead of 1, 2, 3, 4)
\numberwithin{figure}{section} % Number figures within sections (i.e. 1.1, 1.2, 2.1, 2.2 instead of 1, 2, 3, 4)
\numberwithin{table}{section} % Number tables within sections (i.e. 1.1, 1.2, 2.1, 2.2 instead of 1, 2, 3, 4)

\setlength\parindent{0pt} % Removes all indentation from paragraphs - comment this line for an assignment with lots of text

%----------------------------------------------------------------------------------------
%    TITLE SECTION
%----------------------------------------------------------------------------------------

\newcommand{\horrule}[1]{\rule{\linewidth}{#1}} % Create horizontal rule command with 1 argument of height

\title{    
\normalfont \normalsize 
\textsc{Differentiable Manifolds} \\ [25pt] % Your university, school and/or department name(s)
\horrule{0.5pt} \\[0.4cm] % Thin top horizontal rule
\huge Problem Set 3: 1.7 \\ % The assignment title
\horrule{2pt} \\[0.5cm] % Thick bottom horizontal rule
}

\author{Joshua Ramette \& Daniel Halmrast} % Your name

\date{\normalsize\today} % Today's date or a custom date

\begin{document}

\maketitle % Print the title

%----------------------------------------------------------------------------------------
%    PROBLEM 1
%----------------------------------------------------------------------------------------

\section*{Problem 1.7}
Problem 1.7: Prove the following: \\
(a) Let $M$ be a differentiable manifold, and $A \subset M$. Fix a topology on $A$. Then, there is at most one differentiable structure on $A$ such that $(A,i)$ is a submanifold.

Suppose that there are two families of coordinate charts $\mathscr{F}_1$ and $\mathscr{F}_2$ on $A$ such that $(A,i)$ is a submanifold of $M$. We will show that these two families are compatible, and by the maximality of the differentiable structure, they share the same differentiable structure. \\

Let $A_1$ be the manifold $A$ under $\mathscr{F}_1$, and $A_2$ be the manifold $A$ under $\mathscr{F}_2$. 
Then, by theorem 1.32, there exists a unique mapping $\psi_0 : A_1 \to A_2$ such that $i\circle \psi_0 = i$, 
where $i$ is the inclusion map that makes $A$ a submanifold. 
By this relation, however, it must be that $\psi_0 = id$. 
Since $A_1$ and $A_2$ have the same topology, $id$ is a homeomorphism, and by 1.32a, $\psi_0$ is $C^{\infty}$. \\
So, let $(U_1, \phi_1)$ be a coordinate chart on $A_1$, and $(U_2, \phi_2)$ a coordinate chart on $A_2$. 
By the existence of $\psi_0$, the map $\phi_2\circle\psi_0\circle\phi_1^{-1}$ exists and is $C^{\infty}$. 
Furthermore, since $\psi_0 = id$, $\phi_2\circle\psi_0\circle\phi_1^{-1} = \phi_2\circle\phi_1^{-1}$ and is $C^{\infty}$. 
Thus, since this holds for arbitrary coordinate charts on $A_1$ and $A_2$, $\mathscr{F}_1$ is compatible with $\mathscr{F}_2$. \\

Thus, if $A$ has a differentiable structure that makes $(A,i)$ a submanifold, it is unique. 
\\
\\

(b) Let $A \subset M$. If in the relative topology, $A$ has a differentiable structure such that $(A,i)$ is a submanifold of $M$, then $A$ has a unique submanifold structure such that $(A,i)$ is a submanifold of $M$.

Let $A_r$ be $A$ under the relative topology with a differentiable structure such that it is a submanifold under the inclusion map $i_r$.
Then, let $A_t$ be $A$ with any manifold structure that makes it a submanifold under the inclusion map $i_t$. 
To show that $A_r$ is the unique manifold structure on $A$ that makes it a submanifold, we will show that $A_r$ and $A_t$ are diffeomorphic to each other. Furthermore, the diffeomorphism will be the identity map, and the two spaces are actually equal.\\

By theorem 1.32, there exists a unique $\psi_0: A_t \to A_r$ such that $i_r \circle \psi_0 = i_t$.
We will show that such a $\psi_0$ is a diffeomorphism. $\psi_0$ is automatically $C^{\infty}$ since $i_r$ is an imbedding (theorem 1.32b). 
By the result of problem 6, if $\psi_0$ is bijective, and $d\psi_0$ is everywhere nonsingular, then $\psi_0$ is a diffeomorphism.\\
$\psi_0$ is bijective, since $\psi_0 = i_r^{-1} \circle i_t$, which as a set operation is bijective. 
($i_t$ is bijective onto its range, and $dom(i_r^{-1} = ran(i_t)$).\\

To see that $d\psi_0$ is everywhere nonsingular, note that $d\psi_0 = d(i_r^{-1} \circle i_t)$. Since $i_r^{-1}$ is a homeomorphism, and $i_r$ is an immersion, $di_r^{-1}$ is nonsingular. Similarly, since $i_t$ is an immersion, $di_t$ is nonsingular.\\

Therefore, $\psi_0$ is a diffeomorphism. Since $\psi_0 = i_r^{-1} \circle i_t$, which is the identity, the spaces are actually equal. 

Thus, the manifold structure on $A$ is unique.
%----------------------------------------------------------------------------------------

\end{document}
