%%%%%%%%%%%%%%%%%%%%%%%%%%%%%%%%%%%%%%%%%
% Short Sectioned Assignment
% LaTeX Template
% Version 1.0 (5/5/12)
%
% This template has been downloaded from:
% http://www.LaTeXTemplates.com
%
% Original author:
% Frits Wenneker (http://www.howtotex.com)
%
% License:
% CC BY-NC-SA 3.0 (http://creativecommons.org/licenses/by-nc-sa/3.0/)
%
%%%%%%%%%%%%%%%%%%%%%%%%%%%%%%%%%%%%%%%%%

%----------------------------------------------------------------------------------------
%	PACKAGES AND OTHER DOCUMENT CONFIGURATIONS
%----------------------------------------------------------------------------------------

\documentclass[paper=a4, fontsize=11pt]{scrartcl} % A4 paper and 11pt font size

\usepackage[T1]{fontenc} % Use 8-bit encoding that has 256 glyphs
\usepackage{fourier} % Use the Adobe Utopia font for the document - comment this line to return to the LaTeX default
\usepackage[english]{babel} % English language/hyphenation
\usepackage{amsmath,amsfonts,amsthm} % Math packages
\usepackage{mathrsfs}

\usepackage{lipsum} % Used for inserting dummy 'Lorem ipsum' text into the template

\usepackage{sectsty} % Allows customizing section commands
\allsectionsfont{\centering \normalfont\scshape} % Make all sections centered, the default font and small caps

\usepackage{fancyhdr} % Custom headers and footers
\pagestyle{fancyplain} % Makes all pages in the document conform to the custom headers and footers
\fancyhead{} % No page header - if you want one, create it in the same way as the footers below
\fancyfoot[L]{} % Empty left footer
\fancyfoot[C]{} % Empty center footer
\fancyfoot[R]{\thepage} % Page numbering for right footer
\renewcommand{\headrulewidth}{0pt} % Remove header underlines
\renewcommand{\footrulewidth}{0pt} % Remove footer underlines
\setlength{\headheight}{13.6pt} % Customize the height of the header

\numberwithin{equation}{section} % Number equations within sections (i.e. 1.1, 1.2, 2.1, 2.2 instead of 1, 2, 3, 4)
\numberwithin{figure}{section} % Number figures within sections (i.e. 1.1, 1.2, 2.1, 2.2 instead of 1, 2, 3, 4)
\numberwithin{table}{section} % Number tables within sections (i.e. 1.1, 1.2, 2.1, 2.2 instead of 1, 2, 3, 4)

\setlength\parindent{0pt} % Removes all indentation from paragraphs - comment this line for an assignment with lots of text

%----------------------------------------------------------------------------------------
%	TITLE SECTION
%----------------------------------------------------------------------------------------

\newcommand{\horrule}[1]{\rule{\linewidth}{#1}} % Create horizontal rule command with 1 argument of height

\title{	
\normalfont \normalsize 
\textsc{Differentiable Manifolds} \\ [25pt] % Your university, school and/or department name(s)
\horrule{0.5pt} \\[0.4cm] % Thin top horizontal rule
\huge Problem Set 2: 1.5-1.6 \\ % The assignment title
\horrule{2pt} \\[0.5cm] % Thick bottom horizontal rule
}

\author{Joshua Ramette \& Daniel Halmrast} % Your name

\date{\normalsize\today} % Today's date or a custom date

\begin{document}

\maketitle % Print the title

%----------------------------------------------------------------------------------------
%	PROBLEM 1
%----------------------------------------------------------------------------------------

\section*{Problem 1.5}
Problem 1.5: Prove the following: \\

(a) If $(U, \phi)$ and $(V, \psi) \in \mathscr{F}$, then $\widetilde{\psi} \circ \tilde{\phi}^{-1}$ is $C^{\infty}$. It is easily seen that $\tilde{\phi} = \phi \times d\phi$. Then the composition is $\widetilde{\psi} \circ \tilde{\phi}^{-1} = (\psi \circ \phi^{-1}, d(\phi) \circ d(\psi)^{-1}) = (\psi \circ \phi^{-1}, d(\phi \circ \psi^{-1}))$ using the chain rule on differentials. $\psi \circ \phi^{-1}$ is $C^{\infty}$ as the composition of $C^{\infty}$ functions, and the differential preserves the $C^{\infty}$ property of $\psi \circ \phi^{-1}$, so $\widetilde{\psi} \circ \tilde{\phi}^{-1}$ is $C^{\infty}$. \\

(b) The collection $\{\tilde{\phi^{-1}}(W): W$ open in $\mathbb{R}^{2d}, (U, \phi) \in  \mathscr{F} \}     $ forms a basis for a topology on $T(M)$ which makes $T(M)$ into a 2d-dimensional, second countable, locally Euclidean space. Locally Euclidean and 2d-dimensional properties are obvious from the definition of $\tilde{\phi}$ since it defines a homeomorphism between $T(M)$ and $\mathbb{R}^{2d}$. To show it is a basis, notice that $T(M)$ has the initial topology generated by $<\tilde{\phi}>$ for all $(U, \phi) \in \mathscr{F}  $ which has the canonical subbasis $\{\tilde{\phi^{-1}}(W): W$ open in $\mathbb{R}^{2d}, (U, \phi) \in  \mathscr{F} \} $. To show this a basis we have to show it contains its finite intersections. Let $\tilde{\phi_1^{-1}}(W_1)$ and $\tilde{\phi_2^{-1}}(W_2)$ be elements of the subbasis. Then 
\[
\tilde{\phi_1^{-1}}(W_1) \cap \tilde{\phi_2^{-1}}(W_2) = \tilde{\phi_1^{-1}}(\tilde{\phi_1}(\tilde{\phi_1^{-1}}(W_1) \cap \tilde{\phi_2^{-1}}(W_2))
\]
and then define $\tilde{\phi_3^{-1}}(W_3)$ to be $\tilde{\phi_1^{-1}} $ restricted to $\tilde{\phi_1}(\tilde{\phi_1^{-1}}(W_1) \cap \tilde{\phi_2^{-1}}(W_2)$. $\tilde{\phi_3^{-1}}$ is defined since it is just a restriction of $\tilde{\phi_1^{-1}}$ and it is in the canonical subbasis since $\tilde{\phi_3^{-1}}(W_3) = \tilde{\phi_1^{-1}}(W_3)$. \\

To show it is second countable, consider a countable basis $B_C$ in $\mathbb{R}^{2d}$. Let $\{ (U_{\alpha}, \phi_{\alpha}) \}_{\alpha}$ be a differentiable structure on $M$. Using a partition of unity we can construct a countable subset of $\{ (U_{\alpha}, \phi_{\alpha}) \}_{\alpha}$ that charts the whole manifold. Then, for each $\phi_i$ in this countable subset there exists a $\tilde{\phi_i}$ such that $\{ \tilde{\phi_i ^{-1}}(b_j) \}$ for each $b_j \in B_C$ is a countable basis.

(c) It is clear that the collection covers the whole space $T(M)$ since the collection $\{\tilde{\phi^{-1}}(\mathbb{R}^{2d}) \} $ for each $\phi \in \mathscr{F}$ covers the whole space.

\section*{Problem 1.6}
Problem 1.6: Prove that if $\psi$ is one-to-one, onto, and everywhere non-singular, then $\psi$ is a diffeomorphism. \\

Suppose for a contradiction that $\psi$ is not a diffeomorphism. Then, by contrapositive of Cor. A of Inverse Function Theorem, $d\psi$ is not an isomorphism on a neighborhood of a point $m$. But then, since $d\psi$ is everywhere nonsingular, in particular at $m$, then $d\psi$ must not be surjective at $m$. But if $d\psi$ is not surjective at $m$ and $d\psi$ is a linear transformation between vector spaces, dim($M$) < dim($N$). Call $p$ = dim($M$) and $d$ = dim($N$). Then, let $(U, \phi)$ be a coordinate system on $N$ such that $\phi(U) = \mathbb{R}^{d}$.  Then, range$(\phi \circ \psi) = \mathbb{R}^{d}$ as well since $\psi$ is onto. \\

To show this is a contradiction we show that the range of $(\phi \circ \psi)$ has measure zero in $\mathbb{R}^{d}$. First, there exists a countable collection of coordinate charts $(V_i, \tau_i)$ that cover $M$ by the Lindelhof property of second countable spaces. Then, we consider $f: \mathbb{R}^{p} \to M \to \mathbb{R}^{d}$ defined as $f_i = \psi \circ \psi \circ \tau_i ^{-1} $. Note that each $f_i$ is $C^1$ which implies that range of each $f_i$ has measure zero. Because each $f_i$ factors through $V_i \subset M$, then, $\phi \circ \psi (V_i)$ has measure zero for each $V_i$. \\

 Now consider $\phi \circ \psi (M)$. Breaking $M$ up into its countable cover of $V_i$'s,
 \[
 \phi \circ \psi (M) = \phi \circ \psi (\bigcup_{i = 1} ^{\infty} V_i) = \bigcup_{i = 1} ^{\infty} \phi \circ \psi (V_i) \]
 
 Then,
 \[ 
 \mu (\phi \circ \psi (M)) = \mu (\bigcup_{i = 1} ^{\infty} \phi \circ \psi (V_i)) \leq  \bigcup_{i = 1} ^{\infty} \mu( \phi \circ \psi (V_i)) = \sum 0 = 0
 \]
 
 Therefore, the measure of the range is zero, so the range cannot be all of $\mathbb{R}^{d}$.




%----------------------------------------------------------------------------------------

\end{document}